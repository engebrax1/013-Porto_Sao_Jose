\chapter{Mapeamento, com identificação e caracterização das áreas de preservação permanente incidentes sobre o imóvel(banhados, cursos d’água, nascentes, reservatórios artificiais de água, lagos, lagoas, topos de morros e montanhas,dunas, locais de refúgio ou reprodução de aves migratórias ou da fauna ameaçada de extinção);}
\chapter{Relatório Fotográfico atualizado e representativo do terreno proposto}
\chapter{Levantamento Planialtimétrico do imóvel proposto, em escala adequada, contendo curvas de nível (isolinhas)equidistantes de 1 metro, demarcando;}
%3.3.1 - Polígono limite do terreno com sistema urbanístico projetado, com aprovação preliminar do orgão competente domunicípio;
%3.3.2 - Recursos hídricos e seus respectivos níveis máximos normais (cotas máximas de inundação/cheia);
%3.3.3 - Áreas de preservação permanente (app);
%3.3.4 - Locação, em planta ou mapa, dos pontos onde foram tomadas as fotografias do relatório fotográfico, indicando adireção apontada;
%3.3.5 - Locação, em planta ou mapa, dos pontos dos testes de permeabilidade do solo;
%3.3.6 - Locação, em planta ou mapa, dos pontos de sondagem do perfil do solo.
%3.3.7 - Mapa de Isodeclividades do relevo;
%3.3.8 - Aerofoto / imagem de satélite com delimitação da área prevista para o empreendimento;
\chapter{ANOTAÇÃO DE RESPONSABILIDADE TÉCNICA}
%ART Todos os documentos (laudos, testes, plantas, levantamentos, informações, etc.) devem ser encaminhados comassinatura do técnico responsável habilitado, constando o nome, qualificação, registro profissional, endereço e telefonepara contato, com emissão de ART devidamente registrada no conselho de classe correspondente.
