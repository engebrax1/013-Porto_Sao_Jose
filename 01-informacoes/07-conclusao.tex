\subsection{Conclusão do profissional responsável pelo laudo sobre o uso da área, com identificação dos impactos ambientais que resultarão da implantação do empreendimento sobre o meio físico}
%
%Considerando os seguintes fatores:
%\begin{itemize}
%	\item Declividade
%
%Declividade do terreno inferior à 6%
%
%\item Profundidade de lençol freático
%
%Nível de água subterrânea superior à 5 m de profundidade
%
%\item Coeficiente de infiltração no solo
%
%Os testes de percolação indicaram um alto coefieciente de  infiltração no solo.
%
%\end{itemize}
%O terreno é apropriado para edificações residencias de até dois andares com soluções individuais (fossa séptica e sumidouro) para o esgotamento sanitário.

%Na conclusão inserir informações como:
% 1) tipo de solo 
%2) nº sondagens 
%3) nível do lençol 
%4) nº percolação 
%5) coef. médio

Localmente o solo predominante é o latossolo vermelho distrófico e apresenta uma textura arenosa média.
Considerando o resultado das 12 sondagens, a ausência de água
subterrânea até 5 m,
e considerando que
a média dos valores dos 18 testes de percolação foi de 101,6 $\frac{l}{m^2\cdot\,dia}$  
conclui-se que
área objeto deste é apropriada para edificações residenciais  com soluções individuais
(fossa séptica e sumidouro) para o esgotamento sanitário.