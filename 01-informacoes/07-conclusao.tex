\subsection{Conclusão do profissional responsável pelo laudo sobre o uso da área, com identificação dos impactos ambientais que resultarão da implantação do empreendimento sobre o meio físico}
%
%Considerando os seguintes fatores:
%\begin{itemize}
%	\item Declividade
%
%Declividade do terreno inferior à 6%
%
%\item Profundidade de lençol freático
%
%Nível de água subterrânea superior à 5 m de profundidade
%
%\item Coeficiente de infiltração no solo
%
%Os testes de percolação indicaram um alto coefieciente de  infiltração no solo.
%
%\end{itemize}
%O terreno é apropriado para edificações residencias de até dois andares com soluções individuais (fossa séptica e sumidouro) para o esgotamento sanitário.

Considerando o resultado das sondagens, a ausência de água
subterrânea até 5 m,
e considerando que
nenhum teste apontou solo impermeável (C$\leq$20 l/m$^2$.dia), 
ou semi-impermeável, conforme
apresentado na \cref{tab:results},  
conclui-se que
área objeto deste é apropriada para edificações residenciais  com soluções individuais
(fossa séptica e sumidouro) para o esgotamento sanitário.