\section{Informações Sobre o Meio Físico}

\subsection{Descrição geológica, aspectos geotécnicos quanto à estabilidade do terreno, tipologia e grau de compactação do solo para o uso proposto, especificando tipo e altura de cada camada até a profundidade de 5 metros, com marcação de altura do nível freático, quando este ocorrer. Deverão ser executados furos de sondagem distribuídos na área que efetivamente será ocupada, sendo, no mínimo 4 furos para áreas até 1,0 hectare. Para áreas acima de 1,0 ha deverão ser acrescentados mais um furo a cada 2,0 hectares}

\subsubsection{geologia}
Geologicamente, a região encontra-se inserida na Formação Caiuá que é uma das quatro formações que constituem o Grupo Bauru. Os arenitos do Grupo Caiuá, recobrem a maior parte da área e estão dispostas discordantes sobre unidades do Grupo São Bento e em contato transicional com a Formação Adamantina (Grupo Bauru) que apresentam cores marrom avermelhado a arroxeado, mais pálido para a última, características de \textit{red beds}.

Superposta ao Grupo Caiuá ocorre latossolos de textura arenosa, fina a média, que normalmente apresentam alta estabilidade e baixo risco de erosão, contudo, os latossolos do noroeste do Paraná apresentam maior teor de areia (70 a 90\%) e por este motivo, são mais susceptíveis à erosão.

\subsubsection{geomorlogia}

\subsubsection{pedologia}

\subsubsection{interferências antropogênicas (aterros e cortes)}

No local não foi observado a presença de cortes e aterros.

\subsubsection{condições naturais de estabilidade do terreno}

\subsubsection{tipologia e grau de compactação do solo}

Em torno do local das sondagens há pouca declividade,
indicando um terreno estável.

Foram executadas 12 sondagens mecanizadas no dia
01/02/2022, um sábado ensolarado, nos pontos
apresentados na \cref{tab:sondspec}.

\begin{table}[htb!]
	\renewcommand{\multirowsetup}{\centering}
	\renewcommand\tabularxcolumn[1]{m{#1}}
	\centering
	\caption{Sondagens realizadas}
	\label{tab:sondspec}
	\begin{tabularx}{\textwidth}{l X X X X X}
		\toprule
		\multirow{3}{*}{\textbf{Sondagem}} & 
		\multicolumn{3}{c}{\textbf{Profundidade [m]}} &
		\multicolumn{2}{c}{\textbf{Coordenadas [UTM]}}\\
		\cmidrule(r{1em}){2-4}
		\cmidrule(l{1em}){5-6}
		&
		\multicolumn{1}{c}{\textbf{Atingida}} &
		\multicolumn{1}{c}{\textbf{Rocha alterada}} &
		\multicolumn{1}{c}{\textbf{Nível de água}} &
		\multicolumn{1}{c}{\textbf{Longitude [m E]}} &
		\multicolumn{1}{c}{\textbf{Latitude [m S]}} \\
		\midrule
SM  1 & 5 m & - & - & 276.769 & 7.485.330\\
SM  2 & 5 m & - & - & 276.714 & 7.485.387\\
SM  3 & 5 m & - & - & 276.649 & 7.485.451\\
SM  4 & 5 m & - & - & 276.563 & 7.485.517\\
SM  5 & 5 m & - & - & 276.691 & 7.485.593\\
SM  6 & 5 m & - & - & 276.767 & 7.485.548\\
SM  7 & 5 m & - & - & 276.823 & 7.485.485\\
SM  8 & 5 m & - & - & 276.911 & 7.485.473\\
SM  9 & 5 m & - & - & 276.986 & 7.485.532\\
SM 10 & 5 m & - & - & 276.840 & 7.485.619\\
SM 11 & 5 m & - & - & 276.877 & 7.485.746\\
SM 12 & 5 m & - & - & 277.145 & 7.485.655\\
		\bottomrule		
	\end{tabularx}
\end{table}

\FloatBarrier
