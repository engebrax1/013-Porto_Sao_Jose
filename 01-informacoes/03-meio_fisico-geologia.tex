\section{Informações Sobre o Meio Físico}

\subsection{Descrição geológica, aspectos geotécnicos quanto à estabilidade do terreno, tipologia e grau de compactação do solo para o uso proposto, especificando tipo e altura de cada camada até a profundidade de 5 metros, com marcação de altura do nível freático, quando este ocorrer. Deverão ser executados furos de sondagem distribuídos na área que efetivamente será ocupada, sendo, no mínimo 4 furos para áreas até 1,0 hectare. Para áreas acima de 1,0 ha deverão ser acrescentados mais um furo a cada 2,0 hectares}

\subsubsection{geologia}
Geologicamente, a região encontra-se inserida na Formação Caiuá que é uma das quatro formações que constituem o Grupo Bauru. Os arenitos do Grupo Caiuá, recobrem a maior parte da área e estão dispostas discordantes sobre unidades do Grupo São Bento e em contato transicional com a Formação Adamantina (Grupo Bauru) que apresentam cores marrom avermelhado a arroxeado, mais pálido para a última, características de \textit{red beds}.

Superposta ao Grupo Caiuá ocorre latossolos de textura arenosa, fina a média, que normalmente apresentam alta estabilidade e baixo risco de erosão, contudo, os latossolos do noroeste do Paraná apresentam maior teor de areia (70 a 90\%) e por este motivo, são mais susceptíveis à erosão.

\subsubsection{geomorlogia}

Segundo ROSS (1992), o relevo terrestre recebe influência estrutural e escultural direta e  indiretamente calcado em considerações de natureza  conceitual, que estão presentes nessas feições em qualquer tamanho e forma. Assim, suas categorias como: tamanhos, idades, gêneses e forma, são passíveis de serem identificados e cartografados separadamente e, portanto, em categorias distintas.

A área de estudo está inserida na região pertencente à Sub-unidade morfoescultural do Planalto de Campo Mourão, a qual pertence à Unidade Morfo-escultural Terceiro Planalto Paranaense.
O Planalto de Campo Mourão é compreendido entre os rios Ivaí, Piquirí e Paraná. Tem altitudes que variam entre 1.150 metros na escarpa da Esperança, declinando para 225 metros no Rio Paraná.

Localmente a altitude varia entre 256 m e 286 m (próximo à rodovia).
A \cref{fig:geomorfologia}, ilustra o relevo local por meio do modelo digital do terreno,
bem como os pontos em que foram realizadas as sondagens.

\begin{figure}[htb!]
	\centering
	\setlength\fboxsep{0pt}
	\setlength\fboxrule{0.75pt}
	\fbox{\includegraphics[width=\dimexpr1.0\textwidth-2\fboxsep-2\fboxrule\relax]{./imagens/mdt.pdf}}
	\caption{Geomorfologia local}%
	\label{fig:geomorfologia}
\end{figure}

\FloatBarrier



\subsubsection{pedologia}
%LVD19 - Latossolo vermelho distrófico
Localmente o solo é um Latossolo Vermelho distrófico
típico, textura arenosa/média, característico da região noroeste do Paraná.

O Latossolo Vermelho Distrófico caracteriza-se por ser um solo muito desgastado, principalmente por sofrer intemperismo químico, o que causa considerável decomposição de componentes minerais, principalmente de origem
caulinítica (MACHADO SÁ, 2014).
O termo Distrófico refere-se ao fato do solo possuir baixa saturação por
bases, inferior a 50\%. Isso significa que este termo é utilizado para definir solos com
características ácidas.

Latossolos são solos profundos, com
boa estrutura e homogeneização, o que significa possuírem resistência a erosões e
pouca diferenciação entre seu horizontes. Estes solos possuem elevada
permeabilidade, conforme apresentado 
na \cref{tab:percl}.

\subsubsection{interferências antropogênicas (aterros e cortes)}

No local não foi observado a presença de cortes e aterros no local.

\subsubsection{condições naturais de estabilidade do terreno}

A partir da metodologia de Ross (1994), utilizada para a
definição da fragilidade ambiental da área de estudo,
considerou-se, como variável importante de análise,
a declividade, além do tipo de solo e do uso do solo.
%(\cref{tab:fragil}).

Outras características geofísicas levadas em consideração
são apresentadas na \cref{tab:estab}.

\begin{table}[htb!]
	\centering
	\caption{Hierarquização da vulnerabilidade por horizontes diagnósticos
		de subsuperfície}
	\label{tab:estab}
	\begin{tabularx}{\textwidth}{@{\hspace{1cm}}Xlcc}	
		\toprule
		\bfseries Fator & 
		\bfseries Grau de vulnerabilidade &
		\bfseries Peso&
		\multicolumn{1}{c}{\bfseries Característica Local} \\
		\midrule
		\multicolumn{3}{l}{\bfseries Pedologia$\footnotesize^1$}\\\cmidrule{1-1}
		Neossolo  (litólico e quartzarênico) & Muito Alta & 5 & \Square\\
		Argissolo   & Intermediária&3 & \Square\\
		Cambissolos & Intermediária &3& \Square\\
		Espodossolo & Muito Alta &5& \Square\\
		Gleissolo   & Muito Alta &5& \Square\\
		Latossolo   & Muito Baixa &1& \Square\\
		Nitossolo   & Baixa &2& \CheckedBox\\
		Organossolo & Muito Alta &5& \Square\\
		
		\multicolumn{3}{l}{\bfseries Textura$\footnotesize^2$}\\\cmidrule{1-1}
		Argilosa & Muito Baixa  &  1& \CheckedBox\\
		Argilosa Média & Baixa  & 2& \Square\\
		Média e/ou Siltosa & Intermediária  & 3& \Square\\
		Arenosa / Média  &  Muito Alta & 5& \Square\\
		Arenosa  & Muito Alta  & 5& \Square\\
		
		\multicolumn{3}{l}{\bfseries Declividade$\footnotesize^3$}\\\cmidrule{1-1}
		Inferior a 5\% & Muito Baixa & 1& \Square\\
		5 a 15\% & Baixa & 2& \CheckedBox\\
		15 a 30\% & Intermediária & 3& \Square\\
		30 a 45\% & Alta & 4& \Square\\
		Superior a 45\% & Muito Alta & 5& \Square\\
		\midrule
		\multicolumn{3}{c}{\bfseries Total} & \bfseries7\\
		
		\bottomrule
		%		\multicolumn{3}{l}{\footnotesize$^1$ vide \cref{fig:pedologia}}\\
		%		\multicolumn{3}{l}{\footnotesize$^2$ vide \aref{chap:sondg}}\\		
		%		\multicolumn{3}{l}{\footnotesize$^3$ vide \cref{fig:mdt2}}\\		
	\end{tabularx}
\end{table}

\FloatBarrier

Devido a baixa pontuação nos fatores apresentados, pode-se afirmar que o terreno possui alta estabilidade.

\subsubsection{tipologia e grau de compactação do solo}

Em torno do local das sondagens há pouca declividade,
indicando um terreno estável.

Foram executadas 12 sondagens mecanizadas no dia
01/02/2022, um sábado ensolarado, nos pontos
apresentados na \cref{tab:sondspec}. 

\begin{table}[htb!]
	\renewcommand{\multirowsetup}{\centering}
	\renewcommand\tabularxcolumn[1]{m{#1}}
	\centering
	\caption{Sondagens realizadas}
	\label{tab:sondspec}
	\begin{tabularx}{\textwidth}{l X X X X X}
		\toprule
		\multirow{3}{*}{\textbf{Sondagem}} & 
		\multicolumn{3}{c}{\textbf{Profundidade [m]}} &
		\multicolumn{2}{c}{\textbf{Coordenadas [UTM]}}\\
		\cmidrule(r{1em}){2-4}
		\cmidrule(l{1em}){5-6}
		&
		\multicolumn{1}{c}{\textbf{Atingida}} &
		\multicolumn{1}{c}{\textbf{Rocha alterada}} &
		\multicolumn{1}{c}{\textbf{Nível de água}} &
		\multicolumn{1}{c}{\textbf{Longitude [m E]}} &
		\multicolumn{1}{c}{\textbf{Latitude [m S]}} \\
		\midrule
SM  1 & 5 m & - & - & 276.769 & 7.485.330\\
SM  2 & 5 m & - & - & 276.714 & 7.485.387\\
SM  3 & 5 m & - & - & 276.649 & 7.485.451\\
SM  4 & 5 m & - & - & 276.563 & 7.485.517\\
SM  5 & 5 m & - & - & 276.691 & 7.485.593\\
SM  6 & 5 m & - & - & 276.767 & 7.485.548\\
SM  7 & 5 m & - & - & 276.823 & 7.485.485\\
SM  8 & 5 m & - & - & 276.911 & 7.485.473\\
SM  9 & 5 m & - & - & 276.986 & 7.485.532\\
SM 10 & 5 m & - & - & 276.840 & 7.485.619\\
SM 11 & 5 m & - & - & 276.877 & 7.485.746\\
SM 12 & 5 m & - & - & 277.145 & 7.485.655\\
		\bottomrule		
	\end{tabularx}
\end{table}
\FloatBarrier


A \cref{fig:sondfotos} apresenta as fotos tiradas
durante a realização das sondagens.

\import{./figuras_tex/}{figsondagem}
\FloatBarrier
