\addcontentsline{toc}{chapter}{REFERENCIAS}
\chapter*{REFERENCIAS}

\begin{tabularx}{\textwidth}{X}
	
	Carneiro, Maísa Gomes; \textbf{Classificação da Vulnerabilidade Geoambiental e
		Levantamento do Uso e Ocupação do Solo da Bacia do
		Rio Mourão - Paraná}; Campo Mourão  - 2014.
	\\\addlinespace
	
	ASSOCIAÇÃO BRASILEIRA DE NORMAS TÉCNICAS. NBR 15492. Rio de Janeiro, 2007.
	DAEE - Departamento de Águas e Energia Elétrica (1979).\textbf{Estudo de águas subterrâneas - regiões administrativas 10 e 11
		- Presidente Prudente e Marília}. São Paulo
	\\\addlinespace
	
	MACHADO SÁ, M. F. \textbf{Patrimônio natural dos Campos Gerais Paraná.} 1. ed. Capítulo
	6. Ponta Grossa: Editora UEPG, 2014\\\addlinespace
	
	MINISTÉRIO DA SAÚDE. Brasília 2004. \textbf{Manual de Saneamento} 3ª Edição. Brasília/DF\\\addlinespace
	
	MINEROPAR Minerais do Paraná. 2006. \textbf{Mapa Geológico do Estado do Paraná.} Curitiba, MINEROPAR, 1 mapa geológico, escala 1:650.000.
	\\\addlinespace
	
	FERNANDES, L. A., COIMBRA A. M. O Grupo Caiuá (Ks):Revisão estratigráfica e contexto deposicional. \textbf{Revista Brasileira de Geociências}. 09/1994 pg. 164-176.
	\\\addlinespace
	
	Santos L. J. C., Oka-Fiori C., Canali N.E., Fiori A.P., Silveira C.T., Silva 
	J.M.F., Ross J.L.S. 2006. Mapeamento Geomorfológico
	do Estado do Paraná. \textbf{Revista Brasileira de Geomorfologia}, 07:03-1
	\\\addlinespace
	
	Pedron, F. A. et all, Variação das características pedológicas e classificação taxonômica de argissolos derivados de rochas sedimentares. \textbf{Revista Brasileira de Ciência do Solo}, 2012, 9g 1-9, v. 36, n. 1
	\\\addlinespace
	
	ROSS, J. S. \textbf{Registro cartográfico dos fatos geomorfológicos e a questão da taxonomia do relevo}. Rev. Geografia. São Paulo,
	v. 6, p. 17-29, 1992
\end{tabularx}
