\subsection{Descrição e avaliação hidrogeológica local especificando as características físicas dos aquíferos e dos corpos hídricos superficiais no trecho em que se inserem na área do empreendimento (larguras média e máxima, superfície)}

A  área do empreendimento está inserida no Aqüífero Caiuá aflorando em pequena extensão na região do Pontal do Paranapanema e nas proximidades do Rio Paraná, estando encoberto pelas unidades aqüíferas mais jovens no restante dos 31.000 km$^2$ de sua área de ocorrência. Compreende sedimentos arenosos depositados em ambiente fluvial com eventuais interações eólicas mais proeminentes em direção ao topo da sucessão.

O Aqüífero Caiuá tem extensão regional, sendo considerado livre a semiconfinado e contínuo (DAEE, 1979). Condições de semiconfinamento são observadas nas porções onde fácies argilosas típicas da Formação Pirapozinho intercalam-se à Formação Caiuá, ou onde esta última encontra-se sobreposta por outras unidades aqüíferas do Sistema Bauru. O inter-relacionamento entre fácies arenosas e fácies pelíticas, observado nos perfis geofísicos disponíveis, torna o Aqüífero Caiuá heterogêneo e anisotrópico.
Destaca-se que não há nascentes, corpos hídricos que cortam e/ou margeiam a área do empreendimento.
