\subsection{Caracterização do manto de intemperismo com definição dos horizontes pedogenéticos e suas características físicas bem com definição da altura no nível do lençol freático, quando este ocorrer}

\subsubsection*{Espessura do manto de intemperismo}

Após a realização da sondagem mecanizada, constatou-se que o manto de intemperismo (regolito) na área de estudo apresenta uma espessura mínima de 5 m. Vide \cref{tab:sondspec} e
laudos no \aref{chap:sondagens}.

\subsubsection*{Contato com a rocha}

Não houve contato com rocha até a profundidade de 5 m.
Vide \cref{tab:sondspec} e
laudos no \aref{chap:sondagens}.

\subsubsection*{Nível de água}

Não foi encontrado o nível de água até a profundidade de 5 m.
Vide \cref{tab:sondspec} e
laudos no \aref{chap:sondagens}.
