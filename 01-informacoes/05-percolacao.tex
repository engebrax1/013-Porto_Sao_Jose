\subsection{Testes de percolação do solo (ensaios de infiltração), de acordo com NBR 7229/1993 e 13.969/1997, com apresentação dos resultados de cada ensaio (tempos de infiltração e taxa de percolação), indicação da profundidade dacava e locação dos pontos em planta. Informar a data e condições climáticas da época de realização dos testes. A NBR 6.484/2001 sugere 1 sondagem para cada 10.000 m$^2$. Os resultados devem ser interpretados sobre a possibilidade de três (03) ensaios para áreas com até um (01) ha; no mínimo, seis (06) ensaios para áreas entre um (01) ha e até cinco(05) ha. Para áreas acima de cinco (05) ha deverá ser acrescido um (01) ensaio para cada hectare a mais;}

Foram executados 18 testes de percolação na data de 09/02/2022 com céu 
ensolarado. Os dados e resultados constam na 
\cref{tab:results} os quais de acordo com (MINISTÉRIO DA SAÚDE, 2004) 
podem ser classificados
como: 

\begin{itemize}
	\item Rápida
	\item Média
	\item Vagarosa
	\item Semi-impermeável
	\item Impermeável
\end{itemize}

%\newcolumntype{Y}{>{\centering\arraybackslash}X}
\renewcommand{\multirowsetup}{\centering}

\begin{longtable}{@{\hspace{1cm}} cc cc cc}
	
	\caption{Dados e resultados dos testes}\\
	\label{tab:results}\\
	\toprule
	\multirow{2}{*}{\textbf{Teste}} & 
	\multicolumn{1}{c}{\multirow{2}{3.5cm}{\textbf{Coeficiente  de infiltração [l/m$^2$.dia]}}} &
	\multicolumn{1}{c}{\multirow{2}{2.0cm}{\textbf{Permeabilidade [cm/s]}}} &			
	\multicolumn{1}{c}{\multirow{2}{2.4cm}{\textbf{Tempo de rebaixamento}}} & 
	\multicolumn{2}{c}{\textbf{Coordenadas}}
	\\\cmidrule{5-6}
	&&&&
	\textbf{Longitude}&
	\textbf{Latitude}\\
	\midrule
	
	\endfirsthead
	
	\caption{Dados e resultados dos testes (continuação)}\\
	\toprule
	\multirow{2}{*}{\textbf{Teste}} & 
	\multicolumn{1}{c}{\multirow{2}{3.5cm}{\textbf{Coeficiente  de infiltração [l/m$^2$.dia]}}} &
	\multirow{2}{*}{\textbf{Permeabilidade c [cm/s]}} &
	\multicolumn{1}{c}{\multirow{2}{2.5cm}{\textbf{Tempo de rebaixamento}}} & 	
	\multicolumn{2}{c}{\textbf{Coordenadas}}\\\cmidrule{5-6}
	&&&&
	\textbf{Longitude}&
	\textbf{Latitude}\\
	\midrule
	
	\endhead
	
	\midrule
	\multicolumn{6}{r}{\footnotesize\textit{Continua na próxima página}}
	
	\endfoot
	
	\bottomrule
	
	\endlastfoot
	
	\multicolumn{6}{l}{Absorção rápida (C$\geq$90 l/m$^2$.dia)}	\\\cmidrule{1-2}

Cx-15 &152,3& 1,76$\cdot10^{-4}$ & 00 min 43 s &276.882&7.485.678\\ 
Cx-05 &139,3& 1,61$\cdot10^{-4}$ & 01 min 01 s &276.655&7.485.396\\ 
Cx-18 &131,8& 1,53$\cdot10^{-4}$ & 01 min 13 s &276.899&7.485.533\\ 
Cx-10 &125,6& 1,45$\cdot10^{-4}$ & 01 min 24 s &276.981&7.485.483\\ 
Cx-16 &125,6& 1,45$\cdot10^{-4}$ & 01 min 24 s &277.017&7.485.605\\ 
Cx-13 &125,1& 1,45$\cdot10^{-4}$ & 01 min 25 s &276.808&7.485.728\\ 
Cx-01 &123,5& 1,43$\cdot10^{-4}$ & 01 min 28 s &276.789&7.485.328\\ 
Cx-04 &115,3& 1,33$\cdot10^{-4}$ & 01 min 45 s &276.730&7.485.402\\ 
Cx-11 &109,7& 1,27$\cdot10^{-4}$ & 01 min 58 s &276.770&7.485.555\\ 
Cx-08 &107,3& 1,24$\cdot10^{-4}$ & 02 min 04 s &276.726&7.485.485\\ 
Cx-14 &104,6& 1,21$\cdot10^{-4}$ & 02 min 11 s &276.924&7.485.720\\ 


	\multicolumn{6}{l}{Absorção média (60$\leq$C$<$90 l/m$^2$.dia}	\\\cmidrule{1-2}
Cx-06 &85,2& 9,86$\cdot10^{-5}$ & 03 min 15 s &276.609&7.485.496\\ 
Cx-17 &84,5& 9,78$\cdot10^{-5}$ & 03 min 18 s &276.884&7.485.600\\ 
Cx-09 &72,8& 8,42$\cdot10^{-5}$ & 04 min 14 s &276.931&7.485.459\\ 
Cx-03 &63,2& 7,32$\cdot10^{-5}$ & 05 min 15 s &276.720&7.485.307\\ 
Cx-07 &61,9& 7,16$\cdot10^{-5}$ & 05 min 25 s &276.694&7.485.547\\ 

	\multicolumn{6}{l}{Absorção lenta (40$\leq$C$<$60 l/m$^2$.dia}	\\\cmidrule{1-2}
Cx-12 &55,4& 6,41$\cdot10^{-5}$ & 06 min 21 s &276.706&7.485.617\\ 
Cx-02 &45,9& 5,32$\cdot10^{-5}$ & 08 min 10 s &276.747&7.485.277\\ 

	%\multicolumn{6}{l}{Semi impermeável (20$\leq$C$<$40 l/m$^2$.dia}	\\\cmidrule{1-2}
	%Cx-02 & 0,3 m &  33,51  & 12 min 07 s &  423.943  &  7.495.833\\ 
\end{longtable}

O resultado do coeficiente de infiltração para a área objeto deste 
estudo, obtido por meio do tempo de rebaixamento de 15 cm para 14 cm 
de cada teste é ilustrado nas Figuras de \ref{fig:cx1} à 
\ref{fig:cx18}.

\newcommand{\plotcoef}[2]{%
	\begin{tikzpicture}
		\begin{axis}[
			width=0.92\textwidth,
			height=0.35\textheight,  % size of the image
			grid = major,
			grid style = {dashed, gray!30},
			xmin = 10,   % start the diagram at this x-coordinate
			xmax = 150,  % end   the diagram at this x-coordinate
			ymin = 0,   % start the diagram at this y-coordinate
			ymax = 25, % end   the diagram at this y-coordinate
			%		 filter/.code={\pgfmathparse{#1/60}},
			extra x ticks = {#2},
			extra y ticks = {#1},
			extra tick style={% changes for all extra ticks
				tick align=outside,
				tick pos=left,
			},
			extra x tick style={% changes for extra x ticks
				major tick length=1.3\baselineskip,
				/pgf/number format/precision=1,
				/pgf/number format/fixed,
				/pgf/number format/fixed zerofill,
				/pgf/number format/1000 sep={.},
				/pgf/number format/set decimal separator={,}%				
			},
			extra y tick style={% changes for extra y ticks
				major tick length=1.5em,
				y filter/.code = {\pgfmathparse{#1*2}},
				tick label style={rotate=90}
			},
			axis background/.style = {fill=white},
			ylabel = {%
				\begin{minipage}{6cm}
					\centering
					Tempo de infiltração\\%
					(minutos p/ rebaixamento de 1 cm)
			\end{minipage}},
			xlabel = {Coeficiente de infiltração ($l/m^{2}\cdot\,dia$)},
			tick align = outside,]
			\draw[blue, dashed, thick](#2,0) -- (#2,#1); %vertical line
			\draw[blue, dashed, thick](0,#1) -- (#2,#1); %horizontal line
			\addplot[domain=0:150, red, thick] {490/x-2.5}; 
		\end{axis} 
\end{tikzpicture}}

\begin{figure}[htb!]\plotcoef{1.46666666666667}{123.529411764706}\caption{Resultado do teste - Cx-01}\label{fig:cx1}\end{figure}
\begin{figure}[htb!]\plotcoef{8.16666666666667}{45.9375}\caption{Resultado do teste - Cx-02}\label{fig:cx2}\end{figure}
\begin{figure}[htb!]\plotcoef{5.25}{63.2258064516129}\caption{Resultado do teste - Cx-03}\label{fig:cx3}\end{figure}
\begin{figure}[htb!]\plotcoef{1.75}{115.294117647059}\caption{Resultado do teste - Cx-04}\label{fig:cx4}\end{figure}
\begin{figure}[htb!]\plotcoef{1.01666666666667}{139.336492890995}\caption{Resultado do teste - Cx-05}\label{fig:cx5}\end{figure}
\begin{figure}[htb!]\plotcoef{3.25}{85.2173913043478}\caption{Resultado do teste - Cx-06}\label{fig:cx6}\end{figure}
\begin{figure}[htb!]\plotcoef{5.41666666666667}{61.8947368421053}\caption{Resultado do teste - Cx-07}\label{fig:cx7}\end{figure}
\begin{figure}[htb!]\plotcoef{2.06666666666667}{107.299270072993}\caption{Resultado do teste - Cx-08}\label{fig:cx8}\end{figure}
\begin{figure}[htb!]\plotcoef{4.23333333333333}{72.7722772277228}\caption{Resultado do teste - Cx-09}\label{fig:cx9}\end{figure}
\begin{figure}[htb!]\plotcoef{1.4}{125.641025641026}\caption{Resultado do teste - Cx-10}\label{fig:cx10}\end{figure}
\begin{figure}[htb!]\plotcoef{1.96666666666667}{109.701492537313}\caption{Resultado do teste - Cx-11}\label{fig:cx11}\end{figure}
\begin{figure}[htb!]\plotcoef{6.35}{55.3672316384181}\caption{Resultado do teste - Cx-12}\label{fig:cx12}\end{figure}
\begin{figure}[htb!]\plotcoef{1.41666666666667}{125.106382978723}\caption{Resultado do teste - Cx-13}\label{fig:cx13}\end{figure}
\begin{figure}[htb!]\plotcoef{2.18333333333333}{104.626334519573}\caption{Resultado do teste - Cx-14}\label{fig:cx14}\end{figure}
\begin{figure}[htb!]\plotcoef{0.716666666666667}{152.331606217617}\caption{Resultado do teste - Cx-15}\label{fig:cx15}\end{figure}
\begin{figure}[htb!]\plotcoef{1.4}{125.641025641026}\caption{Resultado do teste - Cx-16}\label{fig:cx16}\end{figure}
\begin{figure}[htb!]\plotcoef{3.3}{84.4827586206897}\caption{Resultado do teste - Cx-17}\label{fig:cx17}\end{figure}
\clearpage
\begin{figure}[htb!]\plotcoef{1.21666666666667}{131.838565022422}\caption{Resultado do teste - Cx-18}\label{fig:cx18}\end{figure}


\FloatBarrier



A \cref{fig:cxperc} ilustra os testes realizados.


\import{./figuras_tex/}{figperc}

%O local onde foram tiradas as fotografias
%dos ensaios, é apresentado no \aref{chap:fotografia}.

%\begin{figure}[htb!]
%	\setlength\fboxsep{0pt}
%	\setlength\fboxrule{0.75pt}
%%	\fbox{\includegraphics[width=\dimexpr\textwidth-2\fboxsep-2\fboxrule\relax]{./imagens/fotos_percolacao.pdf}}
%	\caption{Local em que foram tiradas as fotos do ensaio de percolação}%
%	\label{fig:fotosperc}
%\end{figure}

\FloatBarrier
Os resultados obtidos ficaram entre o mínimo de 43 s (Cx-15) e o 
máximo de  8 min e 10 s (Cx-02). Evidencia-se assim que o  lote possui áreas com alto grau de absorção.
