\subsection{Testes de percolação do solo (ensaios de infiltração), de acordo com NBR 7229/1993 e 13.969/1997, com apresentação dos resultados de cada ensaio (tempos de infiltração e taxa de percolação), indicação da profundidade dacava e locação dos pontos em planta. Informar a data e condições climáticas da época de realização dos testes. A NBR 6.484/2001 sugere 1 sondagem para cada 10.000 m$^2$. Os resultados devem ser interpretados sobre a possibilidade de três (03) ensaios para áreas com até um (01) ha; no mínimo, seis (06) ensaios para áreas entre um (01) ha e até cinco(05) ha. Para áreas acima de cinco (05) ha deverá ser acrescido um (01) ensaio para cada hectare a mais;}

Foram executados 3 testes de percolação na data de 27/10/2021 com céu 
ensolarado. Os dados e resultados constam na 
\cref{tab:results} os quais de acordo com (MINISTÉRIO DA SAÚDE, 2004) 
podem ser classificados
como: 

\begin{itemize}
	\item Rápida
	\item Média
	\item Vagarosa
	\item Semi-impermeável
	\item Impermeável
\end{itemize}

%\newcolumntype{Y}{>{\centering\arraybackslash}X}
\renewcommand{\multirowsetup}{\centering}

\begin{longtable}{@{\hspace{1cm}} cc cc cc}
	
	\caption{Dados e resultados dos testes}\\
	\label{tab:results}\\
	\toprule
	\multirow{2}{*}{\textbf{Teste}} & 
	\multirow{2}{*}{\textbf{Profundidade}} &
	\multicolumn{1}{c}{\multirow{2}{3.5cm}{\textbf{Coeficiente  de infiltração [l/m$^2$.dia]}}} &	
	\multicolumn{1}{c}{\multirow{2}{2.5cm}{\textbf{Tempo de rebaixamento}}} & 
	\multicolumn{2}{c}{\textbf{Coordenadas}}
	\\\cmidrule{5-6}
	&&&&
	\textbf{Longitude}&
	\textbf{Latitude}\\
	\midrule
	
	\endfirsthead
	
	\caption{Dados e resultados dos testes (continuação)}\\
	\toprule
	\multirow{2}{*}{\textbf{Teste}} & 
	\multirow{2}{*}{\textbf{Profundidade}} &
	\multicolumn{1}{c}{\multirow{2}{3.5cm}{\textbf{Coeficiente  de infiltração [l/m$^2$.dia]}}} &
	\multicolumn{1}{c}{\multirow{2}{2.5cm}{\textbf{Tempo de rebaixamento}}} & 	
	\multicolumn{2}{c}{\textbf{Coordenadas}}\\\cmidrule{5-6}
	&&&&
	\textbf{Longitude}&
	\textbf{Latitude}\\
	\midrule
	
	\endhead
	
	\midrule
	\multicolumn{6}{r}{\footnotesize\textit{Continua na próxima página}}
	
	\endfoot
	
	\bottomrule
	
	\endlastfoot
	
	\multicolumn{6}{l}{Absorção rápida (C$\geq$90 l/m$^2$.dia)}	\\\cmidrule{1-2}
	Cx-01 & 0,3 m & 106,91 & 02 min 05 s & 400.222 & 7.410.287\\
	Cx-02 & 0,3 m & 142,03 & 00 min 57 s & 400.253 & 7.410.391\\
	Cx-03 & 0,3 m & 125,64 & 01 min 24 s & 400.264 & 7.410.341\\
	
	%\multicolumn{6}{l}{Absorção média (60$\leq$C$<$90 l/m$^2$.dia}	\\\cmidrule{1-2}
	%Cx-07 & 0,3 m &  89,77  & 02 min 58 s &  424.051  &  7.496.011\\ 
	%Cx-23 & 0,3 m &  89,60  & 02 min 58 s &  424.163  &  7.496.326\\ 
	%Cx-18 & 0,3 m &  89,55  & 02 min 58 s &  424.449  &  7.495.923\\ 
	%Cx-21 & 0,3 m &  89,32  & 02 min 59 s &  424.075  &  7.496.125\\ 
	%Cx-19 & 0,3 m &  85,87  & 03 min 12 s &  424.476  &  7.495.816\\ 
	%Cx-22 & 0,3 m &  84,44  & 03 min 18 s &  424.129  &  7.496.227\\ 
	%Cx-20 & 0,3 m &  84,21  & 03 min 19 s &  424.422  &  7.495.857\\ 
	%Cx-13 & 0,3 m &  83,99  & 03 min 20 s &  424.227  &  7.495.806\\ 
	%Cx-16 & 0,3 m &  81,33  & 03 min 31 s &  424.357  &  7.495.855\\ 
	%Cx-24 & 0,3 m &  71,96  & 04 min 19 s &  424.220  &  7.496.365\\ 
	%Cx-01 & 0,3 m &  70,44  & 04 min 27 s &  423.832  &  7.495.935\\ 
	%Cx-08 & 0,3 m &  70,30  & 04 min 28 s &  424.080  &  7.495.927\\ 
	%Cx-10 & 0,3 m &  68,52  & 04 min 39 s &  424.088  &  7.495.877\\ 
	%\multicolumn{6}{l}{Absorção lenta (40$\leq$C$<$60 l/m$^2$.dia}	\\\cmidrule{1-2}
	%Cx-03 & 0,3 m &  59,73  & 05 min 42 s &  424.032  &  7.495.825\\ 
	%Cx-14 & 0,3 m &  59,69  & 05 min 43 s &  424.281  &  7.495.886\\ 
	%Cx-11 & 0,3 m &  56,71  & 06 min 08 s &  424.151  &  7.495.811\\ 
	%Cx-15 & 0,3 m &  56,47  & 06 min 11 s &  424.227  &  7.495.806\\ 
	%\multicolumn{6}{l}{Semi impermeável (20$\leq$C$<$40 l/m$^2$.dia}	\\\cmidrule{1-2}
	%Cx-02 & 0,3 m &  33,51  & 12 min 07 s &  423.943  &  7.495.833\\ 
\end{longtable}

O resultado do coeficiente de infiltração para a área objeto deste 
estudo, obtido por meio do tempo de rebaixamento de 15 cm para 14 cm 
de cada teste é ilustrado nas Figuras de \ref{fig:cx3} à 
\ref{fig:cx2}.

\newcommand{\plotcoef}[2]{%
	\begin{tikzpicture}
		\begin{axis}[
			width=0.92\textwidth,
			height=0.35\textheight,  % size of the image
			grid = major,
			grid style = {dashed, gray!30},
			xmin = 10,   % start the diagram at this x-coordinate
			xmax = 150,  % end   the diagram at this x-coordinate
			ymin = 0,   % start the diagram at this y-coordinate
			ymax = 25, % end   the diagram at this y-coordinate
			%		 filter/.code={\pgfmathparse{#1/60}},
			extra x ticks = {#2},
			extra y ticks = {#1},
			extra tick style={% changes for all extra ticks
				tick align=outside,
				tick pos=left,
			},
			extra x tick style={% changes for extra x ticks
				major tick length=1.3\baselineskip,
				/pgf/number format/precision=2,
				/pgf/number format/fixed,
				/pgf/number format/fixed zerofill,
				/pgf/number format/1000 sep={.},
				/pgf/number format/set decimal separator={,}%				
			},
			extra y tick style={% changes for extra y ticks
				major tick length=1.5em,
				%				y filter/.code = {\pgfmathparse{#1*2}},
				tick label style={rotate=90}
			},
			axis background/.style = {fill=white},
			ylabel = {%
				\begin{minipage}{6cm}
					\centering
					Tempo de infiltração\\%
					(minutos p/ rebaixamento de 1 cm)
			\end{minipage}},
			xlabel = {Coeficiente de infiltração ($l/m^{2}\cdot\,dia$)},
			tick align = outside,]
			\draw[blue, dashed, thick](#2,0) -- (#2,#1); %vertical line
			\draw[blue, dashed, thick](0,#1) -- (#2,#1); %horizontal line
			\addplot[domain=0:150, red, thick] {490/x-2.5}; 
		\end{axis} 
\end{tikzpicture}}

\begin{figure}[htb!]\plotcoef{2.08333333333333}{106.909090909091}\caption{Resultado do teste - Cx-01}\label{fig:cx3}\end{figure}
%\begin{figure}[htb!]\plotcoef{0.95}{142.028985507246}\caption{Resultado do teste - Cx-02}\label{fig:cx1}\end{figure}
%\begin{figure}[htb!]\plotcoef{1.4}{125.641025641026}\caption{Resultado do teste - Cx-03}\label{fig:cx2}\end{figure}



\FloatBarrier

%\begin{figure}[htb!]\plotcoef{2.07566666666667}{107.088220295767}\caption{Resultado do teste - Cx-04}\label{fig:cx4}\end{figure}
%\begin{figure}[htb!]\plotcoef{2.81866666666667}{92.1283529706693}\caption{Resultado do teste - Cx-05}\label{fig:cx5}\end{figure}
%\begin{figure}[htb!]\plotcoef{1.7525}{115.22633744856}\caption{Resultado do teste - Cx-06}\label{fig:cx6}\end{figure}
%\begin{figure}[htb!]\plotcoef{2.95866666666667}{89.7655105031754}\caption{Resultado do teste - Cx-07}\label{fig:cx7}\end{figure}
%\begin{figure}[htb!]\plotcoef{4.46983333333333}{70.30297233315}\caption{Resultado do teste - Cx-08}\label{fig:cx8}\end{figure}\
%\begin{figure}[htb!]\plotcoef{1.90033333333333}{111.355200363609}\caption{Resultado do teste - Cx-09}\label{fig:cx9}\end{figure}
%\begin{figure}[htb!]\plotcoef{4.65116666666667}{68.5202880648845}\caption{Resultado do teste - Cx-10}\label{fig:cx10}\end{figure}
%\begin{figure}[htb!]\plotcoef{6.141}{56.7063997222544}\caption{Resultado do teste - Cx-11}\label{fig:cx11}\end{figure}
%\begin{figure}[htb!]\plotcoef{2.825}{92.018779342723}\caption{Resultado do teste - Cx-12}\label{fig:cx12}\end{figure}
%\begin{figure}[htb!]\plotcoef{3.33433333333333}{83.9856024681483}\caption{Resultado do teste - Cx-13}\label{fig:cx13}\end{figure}
%\begin{figure}[htb!]\plotcoef{5.70866666666667}{59.6930073905628}\caption{Resultado do teste - Cx-14}\label{fig:cx14}\end{figure}
%\begin{figure}[htb!]\plotcoef{6.1765}{56.4743848325938}\caption{Resultado do teste - Cx-15}\label{fig:cx15}\end{figure}
%\begin{figure}[htb!]\plotcoef{3.52483333333333}{81.330050623807}\caption{Resultado do teste - Cx-16}\label{fig:cx16}\end{figure}
%\begin{figure}[htb!]\plotcoef{2.48116666666667}{98.3705289925386}\caption{Resultado do teste - Cx-17}\label{fig:cx17}\end{figure}
%\begin{figure}[htb!]\plotcoef{2.9715}{89.5549666453441}\caption{Resultado do teste - Cx-18}\label{fig:cx18}\end{figure}
%\begin{figure}[htb!]\plotcoef{3.2065}{85.8669937790239}\caption{Resultado do teste - Cx-19}\label{fig:cx19}\end{figure}
%\begin{figure}[htb!]\plotcoef{3.31883333333333}{84.2093203104861}\caption{Resultado do teste - Cx-20}\label{fig:cx20}\end{figure}
%\begin{figure}[htb!]\plotcoef{2.98566666666667}{89.3236920459379}\caption{Resultado do teste - Cx-21}\label{fig:cx21}\end{figure}
%\begin{figure}[htb!]\plotcoef{3.30283333333333}{84.4415084585116}\caption{Resultado do teste - Cx-22}\label{fig:cx22}\end{figure}\clearpage
%\begin{figure}[htb!]\plotcoef{2.96866666666667}{89.6013653541387}\caption{Resultado do teste - Cx-23}\label{fig:cx23}\end{figure}
%\begin{figure}[htb!]\plotcoef{4.309}{71.9635776178587}\caption{Resultado do teste - Cx-24}\label{fig:cx24}\end{figure}


A \cref{fig:cxperc} ilustra os testes realizados.

%\import{./figuras_tex/}{figperc}

O local onde foram tiradas as fotografias
dos ensaios, é apresentado na \cref{fig:fotosperc}.

\begin{figure}[htb!]
	\setlength\fboxsep{0pt}
	\setlength\fboxrule{0.75pt}
%	\fbox{\includegraphics[width=\dimexpr\textwidth-2\fboxsep-2\fboxrule\relax]{./imagens/fotos_percolacao.pdf}}
	\caption{Local em que foram tiradas as fotos do ensaio de percolação}%
	\label{fig:fotosperc}
\end{figure}

\FloatBarrier
Os resultados obtidos ficaram entre o mínimo de 57 s (Cx-02) e o 
máximo de 
2 min e 5 s (Cx-01). Evidencia-se assim que o lote possui um alto coeficiente de 
infiltração.
